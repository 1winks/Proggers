\documentclass{article}
\usepackage[T1]{fontenc}

\title{Opis}

\begin{document}
	
	
	\section*{Dijagram stanja}
	Web aplikacija omogućuje korisnicima pretraživanje ponude knjiga. Kad korisnik dođe na stranicu pretrage tada ima status 'U pretragu'. Korisnik tada može unijeti podatke za pretraživanje. Pri dobivanju rezultata, mora provjeriti je li knjiga ima dostupnih registriranih ponuditelja. Ako nema, onda knjigu se ne smije prikazivati prilikom pretrage u sustavu. Obrnuto, može se. Nakon što završe provjera, prikazuje sve knjige koje se pretražuje. Korisnik može ponoviti ovaj postupak dok ne odustaje od pretraživanje ili izlazak iz stranice pretrage. 
	\section*{Dijagram aktivnosti}
	Proces pretraživanje započinje tako što korisnik odabere opciju pretraživanje knjige i zatim mu se otvara forma za unos podataka.Nakon potvrde upisanih podataka, web aplikacija se spaja na bazu podataka i pokušava dohvatiti podatke iz baze. Prije šaljenje, mora se provjeriti je li knjiga ima dostupnih registriranih ponuditelja. Nakon provjere ako web aplikacija dobije podaci, onda se mora označiti te podatke(ponuditelj) na kartu. Ako korisnik ne stisne te oznake na kartu ili ne dobije podaci, pretraživanje završava. Ako korisnik stisne bilo koje oznake na kartu, web aplikacija pokušava dobiti sve knjige dotičnog ponuditelja iz baze podataka. Nakon uspješno dobivanje podataka, treba prikazati na stranicu.
	\section*{Dijagram komponenti}
	Korisnik pristupa web aplikaciji koristeći web preglednik i sučelje REST API koje su podržane operacije GET i POST protokola HTTP. Usluga za kartu OpenStreetMap komunicira s web aplikacijom isto preko sučelja REST API.Web aplikacija je organizirana modularno. Modul REST odgovara na upite korisnika, komunicira s uslugom za kartu i modulom Server. Modul Server je zadužen za dohvat i obradu podatka iz baze preko sučelja SQL API te komunikaciju s modulom Administrator.
	
	
\end{document}
